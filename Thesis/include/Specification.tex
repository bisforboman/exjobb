\chapter{Modeling \& Specification}

The first two steps in model checking is to translate the sought design into models that can be understood by the model checker and defining the properties specified on the system. This chapter covers the process of these two steps, by first stating some basic concepts and then explaining the iterative process that led to the final results.

\section{Definitions}

First we will explain how a typical design looks, for a \wsn we seek to model. Secondly we will describe in detail how each entity operates. Finally, before explaining the modeling, we will explain the meaning of some key aspects of the work such as \textit{Decisions} and \textit{Over-Collection}.

\subsection{Basic WSN}

A basic \wsn was defined as a starting point for the models. It consisted of a set of collection nodes (referred to as "nodes"), a central server (referred to as "Server") and finally an Environment (the observed source). An illustration of this can be seen in Figure~\ref{fig:basic_wsn}. Which is the basis for all the models described in the thesis. An important note on this setup is that the "environment" here is considered an entity (same as a node or a server). This abstraction was made to simplify the modeling, so the nodes instead of managing a shared resource instead can have the data communicated to them as messages.

\begin{figure}[ht]
    \includegraphics{include/figures/basic_wsn}
    \caption{An illustration of a \wsn}
    \label{fig:basic_wsn}
\end{figure}

%As a start for the formal development process, the project required some "building blocks" \todo{find a better word} to better define certain aspects of a \wsnd. 

%We will henceforth refer to entities in the system as \textbf{actors} in the system.

\subsection{Description of Actors}

A \wsn is built up by several different entities that communicates data between each other. Generally a network consists of multiples of virtually the same entity, e.g. multiple collection nodes, where each of these are running the different instances of the same process. Throughout the thesis these entities will be referred to as \textit{Actors}.

To describe the interaction between two actors in the system, behavior models were used (e.g. Figure~\ref{fig:behav_example}). Where the name of each actor is shown in the boxes at the top. The message channel used between them is shown as the arrows, where the arrow-head points to the actor receiving the message and the contents of the message is referenced above it. Finally the ordering of the messages are in a ascending order from the top, meaning the first message sent is shown furthest to the top of the figure.

\section{Modeling}

This section covers the modeling aspect of the thesis which was one of the main goals of the thesis, first will be discuss some initial assumptions and show some faulty models that were refined to the final results. 

%*what is a model?*
\subsection{Initial Models}

As a starting point, the project began with a minor model of a \wsn. The purpose was to start with a small example containing all relevant parts and then work on expanding this to a more thorough model. A version of this can be seen in appendix \textit{centralized\_decision\_multiple\_nodes\_8jul}. As can be seen in the example, the network passes the data between the actors and when the server triggers the decision to stop, the message is passed through the network and the system shuts down. This captures the entire process of the functionality that was sought from the models. To verify that this functionality was achieved, a correctness property was specified as \textit{When the decision to stop is taken, the system should shut down}. \\ 

This property worked great for a model with one collection node, which was a trivial case since no system relevant for this thesis consists of one collection node. So when more nodes were added, the model checking yielded that the delay when several nodes are gathering data and passing it to the server, the correctness property needed to be more relaxed; which refined the property into \textit{When the decision to stop is taken, the system should \underline{eventually} shut down}. 

\[ \textit{show an image or figure explaining this delay} \]

This was the first refinement done for the models and this model became the initial model for the work, which then was refined into the other models. More refinements will be shown in the section \textbf{Revisions}. Coming chapters will discuss some theoretical work that arose from the this model. The project realized that before continuing the work on the models some generalizations for \wsn needed to be done before working further on the models.

%mention the change with ghost variables to states instead?

\subsection{Variations}

%As a starting point for defining the models, first different architectural choices were considered. This was done to help define different cases of \wsn that could use decisions. The different variations considered were: \\

%Before starting the modeling step, the thesis sought to identify some characteristics for \wsns collecing data, to have some generalizations of \wsns to model instead of targeting a specific system. The variations that were considered were as follows:

During the modeling of the initial models, some assumptions were made for the system. From this we realized that our initial model would only work if the target system had the same architecture, which in many cases wouldn't be true. So we sought a model that should work regardlessly from how the target system should look. Meaning that we sought a model of a system that could be attached to an already installed network, which would help the system achieve Data Minimization. So to expand our idea, and help refine our models, we composed a set of architectural variations for \wsns which our model should work on. These variations were generalizations of \wsns that collected personal data. From this the project hoped to refine a model that should work on each of the different variations. \\

The variations the project settled on were the following:

\begin{itemize}
\item[] \textbf{Centralized or Decentralized decision making} \\ The first choice reflected how much the sensor nodes would analyze the data. Since nodes can have a processing unit, they could potentially analyze the collected data and make a decision on their own.
\item[] \textbf{Conjunctive or Disjunctive decision analysis} \\ The second choice reflected how the decision were processed, if the data from a single data point could trigger a decision or if the decision considered data from multiple entries in it's evaluation.
\item[] \textbf{Centralized or Distributed communication} \\ The final variation reflected how the network communicated, it was considered centralized if all communication were sent through a central unit, such as a server, or if nodes were allowed to communicate independently to each other. 
\end{itemize}


%\section{Initial Modeling} % THIS IS THE SECTION FOR THE ITERATIVE PROCESS

%With this in mind, the modeling began on an initial model was made for of these variations namely; a model with a centralized decision making, disjunctive decision analysis and centralized communication. 

This meant that our already constructed model were a model with \textbf{centralized decision making}, \textbf{disjunctive decision analysis} and \textbf{centralized communication}. More of this will be covered in the design chapter, where the final models are discussed. The following section will explain some intermediate steps that led to the final result.

\section{Revisions}

The process of model checking is an iterative process, with many refinement steps before eventually reaching a final working model. As mentioned in the background, a refinement step for model checking can yield that either the models or the properties need to be changed. This section will show some examples which made refinements to the model, to give some context to the process. \\

\textbf{Introduction of States} \\

% explain the change from the ghost variables to states (since SPIN couldn't verify certain things).

% cause
As can be seen in the initial model, this version handled the verification by using global variables to monitor certain values and making sure the processes were working as intended. An example is the variable \texttt{nodesDone} which kept track on how many nodes were currently actively collecting data. This approach worked initially, but made the models lack certain information. Specifically for the liveness property to know whether a node was collecting data or not. \\ 

% result
This caused the introduction of states in the models instead. This also made the models easier to translate to state machines, since the states were clearly stated in the models.

\subsection{The Network Channel} \\

% explain the introduction of the network channel in the distributed case.

% cause

% result

\subsection{The Liveness Relaxation} \\

% explain when the liveness property was expanded to "bigData_metered" instead.

The liveness property for the system demanded a lot of revisions, which isn't uncommon. A liveness property should capture, loosely explained, that \textit{eventually something good happens}. For this project this was a difficult thing to define, since our solution relied on a state for the system which possibly could never occur in a run. 

% cause

% result

\subsection{The Atomic Requirement} \\

% explain when atomic statements in the models were introduced

% cause

% result


% -------------- ------------------------- old stuff

% FSA -> blablbabla transition

% The first working model can be seen in appendix \textit{cendralized_decision_multiple_nodes_8jul}. 

%This was made as a starting point for other variations and also help define the properties sought of the network.


%This was due to that this variation appeared to be the easiest to implement and analyze, which in turn would help making the other variations and as well formalize the properties sought of the network.

% Furthermore, the project defined the individual components of the \wsn to three actors: \textit{Node}, \textit{Server} and \textit{Environment}.

%Furthermore, the project sought to define a model for the basic \wsn \todo{from 2.1 ref?}, so three actors were defined: \textit{Node}, \textit{Server} and \textit{Environment}.

%Initial models were made for each of the choices except conjunctive decision procedures. 

%This was due to that a conjunctive decision procedure would require a more sophisticated algorithm to analyze the data than the other choices, which would require additional time for just one variation. \todo{is this a relevant chapter?} 

%As suggested when developing formally, I started out with a simple model \todo{only explain how I started out in intro / conclusion} as possible. So the model expects that each actor acts as intended, without malicious users or failing procedures, and that all messages gets through, e.g. no 'lossy' channels. 

%Furthermore, the project sought to define a model for the basic \wsn \todo{from 2.1 ref?}, so three actors were defined: \textit{Node}, \textit{Server} and \textit{Environment}.




\section{Actors}

As mentioned previously, \textit{Actors} represents the entities in the network. This section covers the different actors that reprent the components in our models. 

% describe why the environment is modeled as such
% describe that the environment came later of directly?

% explain how i use them

\subsection{Server Actor}

The server is an actor receiving messages from nodes and storing it for later usage. A server's behavior will vary depending on the structure of the system. If the decision is taken centrally the server will be the one checking for over-collection, otherwise it will be a node. Also if the communication is managed through the server, if the nodes doesn't communicate with each other, the server will act as a repeater for the decision. 

\begin{figure}[ht]
    \includegraphics{include/figures/server_behav}
    \caption{Behavior Model between Server and the Node}
    \label{fig:node_server_behav}
\end{figure}

In Figure \ref{fig:node_server_behav} is the behavior for a system where server makes the decision and nodes doesn't communicate with each other. First, the node sends some data, the server checks for over-collection and replies accordingly. The response will either be a "stop" signaling that over-collection has occurred and the node should stop collecting or it tells it that it can continue collecting. 

In addition to the behavior model, an Finite State Automaton(FSA) were designed for the server (Figure~\ref{fig:server_states}). The initial state being \texttt{Idle\_a}, which the server will stay in until some \texttt{data} is received. "Data" being either \texttt{bigData} or \texttt{smallData}; $ \texttt{data} \in \{ \texttt{smallData}, \texttt{bigData} \} $. The data is received in \texttt{Idle\_a} and \texttt{Idle\_s} and then checked in \texttt{Answ.} and \texttt{Hold}, meaning only the outgoing transitions from \texttt{Idle\_a} and \texttt{Idle\_s} is incoming data, the other are calculated internally. This also means the server will loop indefinetly between \texttt{Answ} and \texttt{Idle\_a} as long as only smallData is received. When \texttt{bigData} is received the server will enter the \texttt{Idle\_s}-\texttt{Hold} loop instead, which denotes the states where the server is requesting the nodes to stop collecting more data. 

%This behavior can be described using states as well, as shown in Figure \ref{fig:server_states}. The same notations are used for the messages sent between the actors except "check" is noted as the state named "Waiting".

\begin{figure}[ht]
    \includegraphics{include/figures/server_actor_fsm}
    \caption{Finite State Automata for the Server Actor}
    \label{fig:server_states}
\end{figure}

\subsection{Environment Actor}

The process for the environment actor had two steps:

\begin{enumerate}
\item Generate random data
\item Serve random data to a requesting node 
\end{enumerate}

As mentioned before, the first step is not intuitive for an environment since the observed source isn't randomly varying, but for modeling purposes this is a simplification made to reduce the complexity of the model. In Figure \ref{fig:behav_example} the behavior between a node and the the environment is described.

%Each node requesting data is placed in a FIFO-queue to the environment, meaning the environment will serve the node that's waited the longest during each iteration. 

\begin{figure}[ht]
    \includegraphics[]{include/figures/env_behav}
    \caption{Behavior Model for the Environment}
    \label{fig:behav_example}
\end{figure}

The corresponding FSA for the environment is seen in Figure~\ref{fig:env_states}. The environment will stay in the initial state \texttt{W} (short for "Waiting"), until a node \texttt{meter} it. Then the data is "generated" in \texttt{G} (short for "generate data") and served back to the node.  

%This behavior can be described using states, as shown in Figure \ref{fig:env_states}. Here the process starts in a waiting state and when a node meters it, some data is generated and sent to the node.

\begin{figure}
    \includegraphics{include/figures/environment_actor}
    \caption{FSA for the Environment Actor}
    \label{fig:env_states}
\end{figure}

\subsection{Node Actor}

As seen in the behavior model for the node actor (Figure \ref{fig:node_behav_smallData}), it captures the majority of a typical scenario for the entire system. That is intuitive since the node communicates with both of the other actors of the system and is a intermediate\todo{is this the right word?} part of the system. The scenario is when a node collects data, that doesn't cause a system-change, and forwards it to the server.

\begin{figure}[ht]
    \includegraphics[scale=1]{include/figures/node_behav_smallData}
    \caption{Behavior Model for a Node}
    \label{fig:node_behav_smallData}
\end{figure}

The alternative behavior for system is described in Figure \ref{fig:node_behav_bigData} instead. There the data collected causes the server to make the decision that the node should stop collecting.

\begin{figure}[ht]
    \includegraphics[scale=1]{include/figures/node_behav_bigData}
    \caption{Behavior Model for a Node over-collecting}
    \label{fig:node_behav_bigData}
\end{figure}

This behavior can be described in a FSA, as seen in Figure~\ref{fig:node_states}.\todo{argue why I've kept the states to a minimum here?} The node meters data from the environment and passes it forward to the system. There it waits (noted by the state \texttt{Wait}) for a response before returning to the \texttt{Idle} state.

\begin{figure}[ht]
    \includegraphics{include/figures/node_actor_fsm}
    \caption{FSA for the Node Actor}
    \label{fig:node_states}
\end{figure}



%The environment is a simple procedure, it starts off by generating random data and serving it to a requesting node. This is a modeling simplification, since the node should rather extract the data from the environment rather then be waiting for a response. But this was made to simplify the system and made the environment easier to manage as a shared resource for the nodes. 

%The collection nodes follows the same procedure, where they start by 'requesting' data from the environment. Then the model was split into two different choices depending on the structure of the system. If the system was decentralized, the decision could be taken by a node. Meaning that then the node would first check if the data requested from the environment was below a certain threshold. Then it would send the data along to the server with either a message 'go', meaning that all was fine, or a message 'stop' meaning that over-collection was occurring and it was going to stop collecting and the other nodes should do the same. 

%If the system was a centralized one, the node would simply pass the data forward to the server. Then it would wait for a response to see if it should continue collecting or if it should stop. This describes a typical scenario for the system, so from this a behavior model for each actor can be defined.

%\[ image here \]

%The behavior model of the nodes can explain the behavior of the entire centralized system, since the node 'communicates' with all the involved parts and is the one taking (being effected of) all the actions. What distinguishes the decentralized system and the centralized can be noticed in the behavior model for the nodes, in the state marked 'data retrieved'. Here the node would check the data on it's own and notify the server of the action being taken, but the behavior model still works for both cases.

%This model expects a perfect behavior from all actors involved, since it doesn't take into account bad behavior (e.g. message-loss or malicious usage) which is not unlikely to occur in a WSN. But this isn't meant to be captured by the model either at this stage, but will be considered when later extending the model.  

%*explain what this model doesn't consider and possible variations for the decisions.*

\section{Specification}

This section presents the properties used to verify the system. The process of defining the properties were an iterative approach and several versions were considered, this section only covers the final properties were the result of this process. 

\subsection{Properties}

The properties defined on the network were formulated using \textit{Linear Time Logic} (LTL). This choice came from the fact that LTL were native to SPIN and the models were abstracted to only focus on the relevant parts to the project, LTL could provide a simple and direct specification to that problem.

\subsubsection{Correctness}

The primary property sought of the system was that it was working as intended. This was formulated as a safety correctness property, to ensure that when decision had been taken, the system respond to the by changing its' behavior.

% explain correctness some more

\begin{definition}{}{} 
Safety Correctness

\textit{When the stop-decision is taken, the system should stop collecting.} \\ 

LTL: $\Box (O \rightarrow (\Diamond D)) $ \\
\end{definition}

Where \textbf{O} and \textbf{D} corresponds to the state where the stop-decision is taken and the states where collection is stopped respectively. This captures the sought system change; whenever the system reaches the state O, eventually it will reach state D. An immediate change is not required, therefore the timing\todo{is this the proper word?} is relaxed by the eventually-operator.

\subsubsection{Liveness}

%A liveness property should express that "eventually something good happens" which for this project became that "eventually a node communicates data to the server". This didn't entirely capture what was sought from the system so an additional property was added for "if a data is communicated to the server, eventually the server replies to the node". 

The second property was intuitive for the system since the initial models were constructed in such a way that when data is sent to the server, the first thing the server does it analyze it and respond accordingly depending on what data was sent.\todo{mention it was verified by design?}  

\begin{definition}{}{}
Liveness (sending)

\textit{Eventually a node sends it's data to the server.} \\

LTL: $\Diamond$ Node\_Send \\

Where Node\_Send denotes the state where the node sends the data to the server. \\
\end{definition} 

\begin{definition}{}{}
Liveness (replying)

\textit{If a node sends data to the server, eventually the server replies to the node.} \\

LTL: $\Box ($Node\_Send $\rightarrow \Diamond$ Server\_Reply$)$\todo{server\_reply doesnt exist atm.} \\

Where Node\_Send means the same as previously and Server\_Reply denotes the state where the server responds to the node.
\end{definition}

%Where \textbf{D} and \textbf{O} are the same events as described previously.