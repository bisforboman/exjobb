\chapter{Background}

In this chapter we introduce the basic concepts that are the focus of this thesis, such as \wsn and \textit{Data Minimization}. % we still start by giving some areas to the research into DM and reasons for it.

%\textit{This section should cover some background information to give the reader some background knowledge to what the project has been about that is required knowledge before moving forward.}

\section{Wireless Sensor Network (WSN)}

%describe my idea of a sensor network and how I will look at them for the sake of the thesis. (node/server/environment)

% how it's built up (components)

A \wsn is recent improvement from the traditional sensor networks, made possible by advances in micro-electro-mechanical systems (MEMS) technology making sensor nodes that are smaller, multifunction and cheaper in comparison to previous sensors. Traditional sensors have two ways of being deployed; 1) They were positioned far away from the actual \textit{phenomenon} (e.g. something known by sense perception) which required large sensors using complex techniques to distinguish the targets from surrounding noise. 2) Several sensors were deployed that only performed sensing and their communication topology had to be carefully engineered and they transmitted time series of the data to the central nodes which performed the communication.\wsns on the other hand, is constructed by deploying a large number of sensor nodes close to the phenomenon and their position doesn't need to be engineered or predetermined.\cite{WSNsurvey} 

% give examples of WSNs?


\section{Data Minimization}

% discuss DM from an abstract POV. Social > Legal > Technical, since there isn't much technical articles regarding DM & WSNs. 

As defined by the EDPS (European Data Protection Supervisor); "The principle of data minimization means that a data controller should limit the collection of personal information to what is directly relevant and necessary to accomplish a specified purpose." \cite{website:europa.eu} %discuss privacy?

%They should also retain the data only for as long as is necessary to fulfil that purpose.

%This makes it an important aspect of privacy in data-collecting settings. 

With the world becomming more and more digitalized\todo{if ref is required there are lots of articles for IoT development} collecting personal data from users becomes a growing concern. Today, data processing entities automatically makes decisions based on data analysis which can impact the lives of individuals, which in turn makes the need for protecting their personal data is even greater.\cite{danezis2015privacy} One problem problem which arises from being more connected is that individuals, whose data is being collected, are often unaware of the consequences of the data processing that comes after. Also, legal repercussions for infringement of data protection obligations are usually only the cause of a breach or misuse that has already occured. There are architectures\todo{is this the correct word?} that support this thinking, to require users know more of the data collected, such as Privacy Enchaning Information Management Systems (PE-IMS) that will be discussed later on in this chapter. 

% to have privacy is a fundamental right for everyone. 

% from this discussion extract what DM seeks to prevent?

% maybe this is a too general paragraph that is too short, should be converted to a more arguing paragraph leading up to the more precise definitions.

In a paper by Pfitzmann, Andreas and Hansen and Marit, a combined terminology for the aspects of Data Minimization was defined. The main definitions they used were: \textit{Anonymity}, \textit{Unlinkability}, \textit{Undetectability} and \textit{Unobservability}.\cite{pfitzmann2010terminology} These definitions will help broaden the explanation of data minimization for the sake of this thesis and therefore we will go through them more thoroughly.\todo{implement this} To give these definition some context, we will use the same terminology as in the paper, where the two most important ones for the definitions are \textit{subjects} and \textit{Items of Interest} (IOI). For privacy reasons, being that we wish to maximize it for human beings, subjects are mainly considered as persons. But in reality it can be a legal person or even be a computer. An Item of Interest is a generalization of what can be seen as information. The contents of a message, the name or the pseudonym of a user or even the action of a user sending a message, all of them can be an Items of Interest.


% persons communicating in the system are \textit{subjects} and the information is regarded as \textit{Items of interest} (IOI). % is this obvious? Should IOIs be better defined?

% It should be noted that this reference isn't as carefully finished as other papers. It seems to be a work in progress and it should perhaps be mentioned in the text.

%It should be noted that the definitions are written with regards to an attacker's perspective. 

\begin{itemize}
\item[] \textbf{Anonymity} means that for a subject to have anonymity, it has to be indistinguisable in a set of subjects. Meaning that if a subject has a set of attributes defining the subject, there always has to exist an appropriate set of subjects with potentially the same attributes, in other words: "Anonymity of a subject means that the subject is not identifiable within a set of subjects, the \textit{anonymity set}.", where the anonymity set is the set of all possible subjects. The opposite of anonymity is called \textit{Identifiability}. 

\item[] \textbf{Unlinkability} is the relation between IOIs in the system. If several IOIs become compromised, an attacker should not be able to distinguish whether these IOIs are related or not. The opposite of unlinkability is called \textit{Linkability}.

\item[] \textbf{Undetectability} is an attribute for an IOI. It means that an attacker should not be able to distinguish whether the item exists or not. The opposite of undetectability is called \textit{Detectability}.

\item[] \textbf{Unobservability} is an attribute which is combination of \textit{anonnymity} and \textit{undetectability}. It means that, in regards to an IOI, that neither a subject involed in the IOI or not involved should be aware of the involved subjects. The opposite of unobservability is called \textit{Unobservability}.
\end{itemize}

% discuss these definitions in a setting? todo on this from prev paragraph

%These definitions also have their opposites; \texit{identifiability}, \textit{linkability}, \textit{detectability} and \textit{observability}. 

%TBR: This covers two important aspects of data minimization, the first being that data should only be kept for as long as it is useful for an application and the second being that they should only collect "relevant" data. The latter is more interesting to the project, since the project's aim is to solve part of this problem. %why is this a good a good definition for me?

\section{Modelling Concurrency}

In this section we will go through some different approaches to modelling concurrency. These will give an insight into some other ways of approaching the problem at hand and the chosen approach will be explained the next chapter.

\subsection{Petri Nets}

In 1962, Carl Adam Petri disserted his work on a more graphical way of modelling concurrency, namely using something called \textit{Petri Nets}. A Petri Net is a directed bipartite graph. Each node in the graph represent transitions, shown as bars, or places, which are shown as circles. Connecting the nodes are directed arcs that describe pre- and/or postconditions for the transitions, illustrated as arrows. % show an example of a petri net



\subsection{Process Calculus}

% other ways of modelling concurrent systems

% mention that neither of the described approaches are used and the chosen approach is covered in the next section

\section{Related Work}

\subsection{Data Erasure and Declassification}

% it's different, but try explain the duality of it and DM. mention outside scope

% some background

Another step to consider when ensuring privacy for users having their data collected is the aspect of releasing or removing the data. In many systems, both of these functionalities are required. For this sake, different policies for \textit{erasure} and \textit{declassification} need to be clearly specified so the users' privacy is protected. 

% how it's used (maybe list one or two examples)

In an article by Chong and Myers\cite{chong2005language}, they propose a security policy framework where policies for both can be specified so it suits the desired application. In said framework, one would specify an erasure policy on under which conditions information must be erased. One could also state what policy that would allow data to survive erasure, since information could be allowed to still exist within a system in a restricted form. Secondly, declassification policies would define what policy should be enforced on new information, the conditions under which said information would be declassified and finally the policy it should have after declassification. 

% Declassifying information means that information that is classified as secret ceases to be to restricted. 

This approach covers an important aspect of privacy, namely the managing of personal data. This differs from the focus of the thesis, as we seek to minimize the collection of data to achieve better privacy. % say something more about the difference/similarities 

% how it differs from DM
%This approach to ensuring privacy on personal data is closely related to data minimization, as it covers another aspect of data minimization namely the storing on personal data as long as it's required. 

\subsection{Privacy Enhancing Identity Management Systems (PE-IMS)}

% use definitions in previous section in this section aswell for explanations, this is already done in the article/paper/smth?

In an online setting, it's assumed that people would like to retain their anononymity.\cite{hansen2004privacy} To let users manually control their identities would be a cumbersome process, so instead an automated solution managing this would be preferred. Such a solution can be an Identity Management System (IMS).

An IMS is a system that allows support for "administration of information subjects". An extension of this is Privacy-Enhancing Identity Management Systems (PE-IMS) which supports "active management of personal information" which grants all parties involved flexibility and control over their personal data. A principle used for this is called 'Notice and Choice', a central aspect of data minimization, which means user-controlled linkability of personal data. This puts the responsibility on the user to make informed choices of representing and managing their partial identities.

This allows a user to be as anonymous as they wish, within the predefined limits, since a PE-IMS can be designed to offer any degree of anonymity and linkability. Applications utilizing PE-IMS would specify the range of choices available to the user. Some applications might require some authenticity from the user, e.g. government processes, and in other cases a user could be allowed complete anonymity. By allowing each application different levels of authorization, one can minimize linkability between different communication events and still maximize information exchange while preventing context-spanning profiling. % this section has a lot from the paper.

% --- 
%\textit{compare this to your work}

%\subsection{Security and Privacy for Mobile Electronic Health Monitoring and Recording Systems}
%\todo{Health applications using DM}\cite{barnickel2010security}
%\textit{discuss how healthnet works}
%In a paper from 2010 Barnickel, Karahan and Meyer investigated a mobile health monitoring and data collection system called HealthNet, which is a joint research project of several research groups of RWTH Aachen University. The system consists of a sensor network embedded in a users clothing which collects vital parameters and wirelessly communicates it to the wearer's phone. From there the data is managed, stored and transfered securely to relevant parties, such as medical experts, emergency care and private parties trusted by the wearer. By using the phone the user can control who may access the data. The system is deisgned so that the emergency support can be managed automatically when the vital signs match predefined patterns and not to create an infrastructure among medical experts or health insurance companies. Confidentially and integrity is managed through cryptographic techniques, which prevent attacks on the transmissions and stolen devices. 
%\textit{find the entry point for discussing DM}

%\subsection{Smart City} 
%\label{section:SmartCity}

%\textit{some background into their approach}\cite{li2016privacy}
%\textit{some comparison to this project}
%*discuss smart city's approach in difference to your own*
% ---
