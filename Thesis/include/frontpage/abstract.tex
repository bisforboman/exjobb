Data Minimization in Distributed Applications for More Privacy\\
%A Subtitle that can be Very Much Longer if Necessary\\
JAKOB BOMAN\\
Department of Computer Science and Engineering\\
Chalmers University of Technology \setlength{\parskip}{0.5cm}

\thispagestyle{plain}			% Supress header 
\setlength{\parskip}{0pt plus 1.0pt}

\section*{Abstract}

The presence of connected devices in our environment is increasing. These devices form a network often called Internet of Things (or IoT for short), where everything from light-bulbs to thermostats can be controlled by an app or by another device. These services make a lot of that data available to the end user but also to malicious parties due to the devices leaking more data than intended or by bad design. This puts the end user at risk, violating its privacy and leaking sensitive data. One simple and obvious way to prevent leakages and misuses of personal data is to collect less of this data, a principle known as data minimization. However, this solution is rarely used in practice because of business models relying on personal data harvest on one hand and because of the difficulty to enforce it once it is defined what is actually needed to provide a service.

% KEYWORDS (MAXIMUM 10 WORDS)
\vfill
Keywords: \textit{some keywords will be added here}

\newpage				% Create empty back of side
\thispagestyle{empty}
\mbox{}