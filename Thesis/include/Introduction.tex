\chapter{Introduction}

This chapter will give a brief introduction to the thesis, discuss the motivation for why this thesis is relevant and then also discuss the aims sought to be achieved at the end.

\section{Motivation}

%\textit{discuss privacy}

% why privacy?

% indentity theft, 

% WIKIPEDIA:
% the right to be let alone
% the option to limit the access others have to one's personal information
% secrecy, or the option to conceal any information from others
% control over others' use of information about oneself
% states of privacy
% personhood and autonomy
% self-identity and personal growth
% protection of intimate relationships

%Why should we urge for the need of privacy? \\

%Privacy in itself isn't obvious for some people. One can also argue that governments should have as much information as they want, e.g. to prevent terrorist attacks, while others might value their choice to share only the information they want. The latter of these is the core for this thesis, which also is the core aspect for privacy; let oneself decide which information should be shared. 

%The setting for this is \wsn, which is a commonly used tool today for automatically collecting data. The important part arises from the sensors, e.g. a camera or a temperature meter. They are often designed to gathering as much data as possible, e.g. the camera having as good resolution as possible. Then this camera could be used for mapping an individuals pattern and analyzing it's behaviour, which the individual unwillingly has shared.

%\cite{weber2010internet}

% https://opensignal.com/reports/2015/03/the-privacy-implications-of-mobile-sensor-networks/
%In a speech, given at 4YFN Barcelona on the 2nd of March 2015 by Samuel Johnston, 
  % - mention the lag!

% An important problem with data collection is the aspect of privacy. 

% Maintaining privacy is a multi-faceted issue. In some cases the collected data can be anonymized and in such a way achieve privacy by making it 

% 1. mention privacy
The presence of connected devices in our environment is increasing. These devices form a network often called Internet of Things (or IoT for short), where everything from lightbulbs to thermostats can be controlled by an app or by another device. These services make a lot of that data available to the end user but also to malicious parties due to the devices leaking more data than intended or by bad design. This puts the end user at risk, violating its privacy and leaking sensitive data. 

One simple and obvious way to prevent leakages and misuses of personal data is to collect less of this data, a principle known as data minimisation. However, this solution is rarely used in practice because of business models relying on personal data harvest on one hand and because of the difficulty to enforce it once it is defined what is actually needed to provide a service.

% 2. mention DM
Privacy is utterly important for the development of IoT applications, Miorandi et al. gives several reasons\todo{insert ref}: “The main reasons that makes privacy a fundamental IoT requirement lies in the envisioned IoT application domains and in the technologies used. Healthcare applications represent the most outstanding application field, whereby the lack of appropriate mechanisms for ensuring privacy of personal and/or sensitive information has harnessed the adoption of IoT technologies.“ 

% 3. mention what I want with it ?



\section{Aim}

%The purpose of this thesis is improve privacy in \wsns, in the sense that minimizing the amount of personal data being processed reduces the amount of data being communicated in the network. Different versions of \wsns will be considered and analyzed to reach a general solution as possible. 

%The project also seeks to formalize the meaning of overcollection and what it means in the setting of \wsns collecting personal data. 

This thesis will investigate ways to improve privacy in a special kind of IoT (Internet of Things) devices known as Wireless Sensor Networks (WSN). WSN are networks of autonomous sensors and actuators. The goal to enhance privacy for this kind of devices will be addressed by relying on data minimization. This means the project sought to improve privacy in distributed networks by limiting the amount of personal data being processed.

To achieve this, the project sought to accomplish the following steps:

\begin{itemize}
\item Define Over-Collection, it's relation to Data Minimization and it's meaning with regards to this project. 
\item Construct a model of an example of a \wsn that stops the collection when Over-Collection is occuring.
\item Implement (or generate) a C-implementation from the models and analyze them.
\end{itemize} % C-implementation?

%maybe add some list with goals?

\section{Limitations}

% faulty wsns
The project will not consider faulty behaviors of a \wsn, meaning that the systems and algorithms will work under the assumption that all messages sent are received and all units are working as intended without malfunctions.

% time complexity
Only the result of the data collection will be analyzed in the sought outcome and if time complexity of the algorithm will be an issue for the project, it will not be considered as a failure should it arise. Some analysis will be done but it won't be a main focus to minifor the project.

% data collection
Only the privacy aspects of collecting personal data will be considered throughout the thesis and the aspect of storing and managing it will be outside the scope of this thesis.

% time vs states
Any model properties related to time (in the sense that they can be measured numerically) will be treated on an abstract level or be disregarded.\todo{this section is borrowed from another MT where a quote was also included from Ben-Ari's book from 2008. I thought it was relevant for me aswell} %borrowed from other thesis, referred to Ben-Ari 2008 p. 173.

% attackers
% include that attack vectors and stuff are outside the scope of the project?

%The project seeks to model and verify an overcollecting system and investigate 

%\textit{in this section I will explain what questions the project sought/didn't seek to answer and what problems it did and didn't attempt to solve.}

% define overcollection formally

\section{Thesis Structure}

\textit{saving for later when the thesis has shaped up}

